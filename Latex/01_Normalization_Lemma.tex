\documentclass{article}
\usepackage{amsmath,amsthm,amsfonts}

\newtheorem{definition}{Definition}[section]
\newtheorem{conjecture}{Conjecture}[section]
\newtheorem{lemma}{Lemma}[section]

\begin{document}

\title{Introduction: Modified Collatz Conjecture}
\maketitle

In the realm of discrete dynamical systems, the Collatz conjecture has long stood as a tantalizing enigma. Despite its deceptively simple formulation, it has defied proof and has thus remained one of the most notorious unsolved problems in mathematics. In this paper, we propose a novel reformulation of the Collatz conjecture that may provide new avenues for exploration.

\begin{definition}
Let \(C: \mathbb{N} \rightarrow \mathbb{N}\) be a function defined as follows:
\[
C(x) = 3x + 2^m
\]
where \(m \in \mathbb{N}\) is the largest integer such that \(x/2^m\) is even. This function encapsulates the "constructive" operation of the traditional Collatz function, but also includes the subsequent divisions by 2 until the result is odd.
\end{definition}

\begin{conjecture}
For any \(n \in \mathbb{N}\), the sequence \(n, C(n), C(C(n)), C(C(C(n))), \ldots\) eventually reaches a power of 2.
\end{conjecture}

\begin{lemma}
Let \(x \in \mathbb{N}\) be a positive integer and let \(m \in \mathbb{N}\) be the largest integer such that \(x/2^m\) is even. Then the number \(x' = x / 2^m\) is odd and represents the "normalized" or \(2^0\)-based form of \(x\).
\end{lemma}

\begin{proof}
By the definition of \(m\), the number \(x' = x / 2^m\) is even. However, since \(m\) is the largest integer with this property, it follows that \(x' / 2\) is odd. Therefore, \(x'\) itself must be odd.
\end{proof}

In the subsequent sections of this paper, we will explore the implications of this normalization process for the dynamics of the modified Collatz function and the structure of the sequences it generates.

\end{document}
