


\documentclass{article}
\usepackage{amsmath}
\usepackage{amsthm}

\newtheorem{theorem}{Theorem}

\begin{document}

\begin{theorem}[Bounded Growth Strength]
Let $N_i$ be an odd positive integer with binary representation of the form $(b_1 b_2 \ldots b_k)_2$, where $b_1, b_2, \ldots, b_k$ are binary digits and $b_k = 1$ (since $N_i$ is odd). Then, the growth strength $G(N_i)$ of the binary representation of $N_i$ under the Collatz operation is bounded above by 2 bits per odd iteration.
\end{theorem}

\begin{proof}
Let $N_i$ be an odd positive integer with binary representation $(b_1 b_2 \ldots b_k)_2$. The Collatz operation for an odd integer is defined as $3N_i + 1$. Applying this operation to $N_i$, we have:
\[ 3N_i + 1 = 3(b_1 b_2 \ldots b_k)_2 + 1 \]

In binary arithmetic, multiplying an odd number by 3 is equivalent to left-shifting the number by one position and adding the original number. Therefore, the binary representation of $3N_i$ is given by:
\[ (b_1 b_2 \ldots b_k 0)_2 + (b_1 b_2 \ldots b_k)_2 \]

Adding 1 to this sum, we obtain:
\[ 3N_i + 1 = (b_1 b_2 \ldots b_k 0)_2 + (b_1 b_2 \ldots b_k)_2 + (1)_2 = (c_1 c_2 \ldots c_m 1)_2 \]

where $c_1, c_2, \ldots, c_m$ are binary digits, and the carry generated from the addition of the least significant bits may propagate to the left, potentially causing a change in the most significant bits.

The key observation is that the maximum growth in the number of bits occurs when the most significant bits of $N_i$ are ``11'' (e.g., $(1101)_2$). In this case, the operation results in $(10001)_2$ after the carry propagates. This represents an increase of 2 bits compared to the original number of bits in $N_i$.

Therefore, we can formally conclude that the growth strength $G(N_i)$ is bounded above by 2 bits per odd iteration, as the maximum increase in the number of bits occurs when the head segment is ``11'' and the operation results in $(10001)_2$. This bound holds for all odd positive integers $N_i$ in the sequence.
\end{proof}

\end{document}
