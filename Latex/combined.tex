\documentclass{article}
\usepackage{amsmath}
\usepackage{amssymb}
\usepackage{amsthm}

\theoremstyle{definition}
\newtheorem{definition}{Definition}

\begin{document}

\section{Definitions and Explanation}

\begin{definition}[Constructive Function]
Let \( C \colon \mathbb{N}_{\text{odd}} \to \mathbb{N} \) be the constructive function defined by
\[ C(x) = 3x + 1 \]
for all \( x \in \mathbb{N}_{\text{odd}} \).
\end{definition}

\begin{definition}[Destructive Function]
Let \( D \colon \mathbb{N} \to \mathbb{N}_{\text{odd}} \) be the destructive function defined by iteratively dividing a positive integer \( y \) by 2 until the result is odd. Formally, for \( y = k \cdot 2^m \) where \( k \) is an odd positive integer and \( m \in \mathbb{N} \), the destructive function is defined as
\[ D(y) = k \]
The number of divisions, \( m \), represents the magnitude of the destructive mode.
\end{definition}

\begin{definition}[Collatz Process]
The Collatz process for \( n \in \mathbb{N} \) is a sequence of applications of the constructive function \( C \) and the destructive function \( D \), starting with \( C(n) \) and alternating between \( C \) and \( D \) until reaching the value 1. The process is denoted as \( \mathcal{P}(n) \).
\end{definition}

\begin{definition}[Magnitude of Constructive and Destructive Modes]
Let \( b \colon \mathbb{N} \to \mathbb{N} \) be a function that maps a positive integer to the number of bits in its binary representation. The magnitude of the constructive mode for \( x \in \mathbb{N}_{\text{odd}} \), denoted by \( G(x) \), is given by
\[ G(x) = b(C(x)) - b(x) \]
The magnitude of the destructive mode for \( y \in \mathbb{N} \) with \( y = k \cdot 2^m \), denoted by \( R(y) \), is given by
\[ R(y) = m \]
\end{definition}

\begin{definition}[Mode Oscillation]
The mode oscillation in the Collatz process refers to the alternation between the constructive mode, represented by the function \( C \), and the destructive mode, represented by the function \( D \). The oscillation between these modes serves as a clock for normalizing the \( x \)-axis, providing a uniform means for assessing the magnitudes of construction and destruction. The values of \( G(x) \) and \( R(y) \) represent the magnitudes of the constructive and destructive modes, respectively, for the given positive integers \( x \) and \( y \). The Collatz process \( \mathcal{P}(n) \) can be analyzed in terms of these magnitudes to study the behavior of the sequence.
\end{definition}

\end{document}
\documentclass{article}
\usepackage{amsmath}
\usepackage{amssymb}
\usepackage{amsthm}

\theoremstyle{definition}
\newtheorem{theorem}{Theorem}

\begin{document}

\section{Bounded Growth Strength: Formal Justification}

\begin{theorem}[Bounded Growth Strength]
Let \( N_i \) be an odd positive integer with binary representation of the form \( (b_1 b_2 \ldots b_k)_2 \), where \( b_1, b_2, \ldots, b_k \) are binary digits and \( b_k = 1 \) (since \( N_i \) is odd). Then, the growth strength \( G(N_i) \) of the binary representation of \( N_i \) under the Collatz operation is bounded above by 2 bits per odd iteration.
\end{theorem}

\begin{proof}
Let \( N_i \) be an odd positive integer with binary representation \( (b_1 b_2 \ldots b_k)_2 \). The Collatz operation for an odd integer is defined as \( 3N_i + 1 \). Applying this operation to \( N_i \), we have:
\[ 3N_i + 1 = 3(b_1 b_2 \ldots b_k)_2 + 1 \]

In binary arithmetic, multiplying an odd number by 3 is equivalent to left-shifting the number by one position and adding the original number. Therefore, the binary representation of \( 3N_i \) is given by:
\[ (b_1 b_2 \ldots b_k 0)_2 + (b_1 b_2 \ldots b_k)_2 \]

Adding 1 to this sum, we obtain:
\[ 3N_i + 1 = (b_1 b_2 \ldots b_k 0)_2 + (b_1 b_2 \ldots b_k)_2 + (1)_2 = (c_1 c_2 \ldots c_m 1)_2 \]

where \( c_1, c_2, \ldots, c_m \) are binary digits, and the carry generated from the addition of the least significant bits may propagate to the left, potentially causing a change in the most significant bits.

The key observation is that the maximum growth in the number of bits occurs when the most significant bits of \( N_i \) are ``11'' (e.g., \( (1101)_2 \)). In this case, the operation results in \( (10001)_2 \) after the carry propagates. This represents an increase of 2 bits compared to the original number of bits in \( N_i \).

Therefore, we can formally conclude that the growth strength \( G(N_i) \) is bounded above by 2 bits per odd iteration, as the maximum increase in the number of bits occurs when the head segment is ``11'' and the operation results in \( (10001)_2 \). This bound holds for all odd positive integers \( N_i \) in the sequence.
\end{proof}

\end{document}
