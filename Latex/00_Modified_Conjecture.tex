
\documentclass{article}
\usepackage{amsmath,amsthm,amsfonts}

\newtheorem{definition}{Definition}[section]
\newtheorem{conjecture}{Conjecture}[section]

\begin{document}

\title{Introduction: Modified Collatz Conjecture}
\maketitle

In the realm of discrete dynamical systems, the Collatz conjecture has long stood as a tantalizing enigma. Despite its deceptively simple formulation, it has defied proof and has thus remained one of the most notorious unsolved problems in mathematics. In this paper, we propose a novel reformulation of the Collatz conjecture that may provide new avenues for exploration.

\begin{definition}
Let \(C: \mathbb{N} \rightarrow \mathbb{N}\) be a function defined as follows:
\[
C(x) = 3x + 2^m
\]
where \(m \in \mathbb{N}\) is the largest integer such that \(x/2^m\) is even. This function encapsulates the "constructive" operation of the traditional Collatz function, but also includes the subsequent divisions by 2 until the result is odd.
\end{definition}

\begin{conjecture}
For any \(n \in \mathbb{N}\), the sequence \(n, C(n), C(C(n)), C(C(C(n))), \ldots\) eventually reaches a power of 2.
\end{conjecture}

This reformulation of the Collatz conjecture, which we will refer to as the Modified Collatz Conjecture, provides a new perspective on the problem. It combines the "constructive" and "destructive" steps of the traditional Collatz function into a single operation. This new perspective may yield fresh insights into the underlying dynamics of the Collatz sequence and potentially provide a pathway to a proof of the original conjecture.

In the subsequent sections of this paper, we will explore the properties of the modified Collatz function, investigate its fixed and periodic points, and analyze its behavior using tools from dynamical systems theory, algebraic geometry, and probabilistic number theory.

\end{document}
